\documentclass[11pt]{article}
\usepackage{acl2012}
\usepackage{geometry}
\usepackage{times}
\usepackage{latexsym}
\usepackage{amsmath}
\usepackage{multirow}
\usepackage{url}
\newgeometry{margin=2.85cm}
\DeclareMathOperator*{\argmax}{arg\,max}
\setlength\titlebox{6.8cm}    % Expanding the titlebox

\title{{\small CS224W Final Project} \\ Proposal}
\author{Thomas Dimson \\
  {\tt tdimson@cs.stanford.edu}
  \\\And
  Milind Ganjoo \\
  {\tt mganjoo@cs.stanford.edu}
}
\date{}

\newcommand{\titlecite}[2]{``#1''~\cite{#2}}

\begin{document}
\maketitle

\section{Introduction}

\section{Reaction Paper}
\subsection{Quantifying Influence on Twitter}
In \titlecite{Everyone's an influencer: quantifying influence on twitter}{bakshy2011everyone}, 
some stuff happens.

% ROUGH NOTES
% * Paper observes cascade / diffusion events to quantify "influencers"
% * Defines influencers as those who disproportionally impact the spread of information
% ** We could investigate whether this corresponds to karma, if our network has cascades
% * Methodology: observe the origin of URLs, then track them as they are tweeted by followers
% * Trains model for cascades based on followers, tweets,  friends, past influence. Past influence most predictive
% ** Finds local features (direct followers) are more important than global. Could try to see if karma is influenced
%    by people close by
% * Also tries to find out the role of /content/ in cascades by using mech turk to classify the URLs
% ** Did not get good results, but maybe we can try a few experiments to see if karma distribution changes for topics (e.g. by LDA)
% * Paper issues:
% ** Naturally biased, since it _defines_ influence as this. Karma is directly observable so we
%    can see what actually corresponds to karma

\subsection{Twitterrank}
In \titlecite{Twitterrank: finding topic-sensitive influential twitterers}{weng2010twitterrank}, 
some nifty jazz occurs.
% ROUGH NOTES
% * Paper comes up with a new ranking algorithm it claims corresponds to influence on twitter
% * Claims PageRank ignores the /interest/ of twitter
% * Aggregates tweets of a user into a single document and performs LDA on the resultant docs
% * Idea: followers are likely to have similar probability distributions over topics (
%   as measured by KL-divergence)
% ** Validates this hypothesis on their dataset emphatically
% * Invents a modified version of topic sensitive page rank that has transition probabilities
%   to all nodes affected by their topic similarity.
% * Evaluation: remove follower/target relationship, run algorithm, see if they can predict the relationship
% ** Algorithm performs very well in this task
% * They quantify general influence as a weighted sum over topic-specific influences
% * Take aways:
% ** It seems plausible that karma is influenced by the primary subject area of the poster
% ** We might require multiple models for karma depending on topic
% ** We should validate the hypothesis that repliers and posters tend to have similar topic distributions
% ** It is possible that karma is related to PageRank, and refined with topic sensitive PageRank
% * Paper issues:
% ** Evaluation set was seeded by 1,000 top singaporian twitter users, not randomly. Prone to implicit bias
% ** Only 6748 people in the evaluation set - seems very small

\subsection{False indicators of influence}
In \titlecite{Measuring User Influence in Twitter: The Million Follower Fallacy}{weng2010twitterrank}, 
some additional nifty jazz occurs.
% ROUGH NOTES
% * Paper compares three measures of influence: indegree, retweets and mentios.
% * Observes how popular news topics spread by the three typeso f influential users.
% * Questions the traditional view that only a few select people have an instrinsic quality that makes them influential. i.e anyone can "make it" with the right amount of focus or the right set of circumstances.
% * Take away 1: the three measures mean different things, e.g followers/indegree indicates "popularity", retweets represents content value, and mentions indicate name value. Our objective is similar -- we intend to study what aspect of influence karma score actually measures.
% Note 1: can we sift out users with high karma values but who are just popular because they've been around long, and not necessarily because they're useful? How do you measure "value" of their contributions? Can we study the nature of their relationships with others? (e.g. are they perceived positively by other high-ranked members? or is there a lot of antagonism/arrogance?)
% * Note 2: can we measure evolution of karma with time? Don't want "15 seconds of fame" people who've since then retained their influence, because karma scores are usually static.
% * Methodology: rank correlation coefficient (need to discuss this in a little detail) to compare different measures of influcence
% * Findings: retweets usually prominent among content aggregation services (content value), mentions among celebrities (name value)
% * Second study: does influence change across topic genres? (ie is a person who posts only about one theme more or less influential than a person who posts about everything?) According to them, generalists tend to have more influence, but it would be nice to use our LDA topic stuff to segment users and try performing influence calculations in those separate contexts.



We'll probably cite \titlecite{Topic-sensitive pagerank}{haveliwala2002topic} too. And what kind of a crazy
paper doesn't cite \titlecite{Latent dirichlet allocation}{blei2003latent}?

\section{Project Proposal}
\bibliography{proposal}{} \bibliographystyle{acl2012}

\end{document}
